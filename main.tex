\documentclass[compress]{beamer}

%Importando alguns pacotes
\usepackage{lipsum}
\usepackage{fancyvrb}
\usepackage{ragged2e}
\usepackage{xcolor}
\usepackage{graphicx}
\usepackage{tikz}
\usepackage[alf]{abntex2cite}

%Importando alguns pacotes
\usepackage{appendixnumberbeamer}
\usepackage{booktabs}
\usepackage[scale=2]{ccicons}

%Importando alguns pacotes
\usepackage{pgfplots}
\usepgfplotslibrary{dateplot}

%Importando alguns pacotes e criando um novo comando
\usepackage{xspace}
\newcommand{\themename}{\textbf{\textsc{metropolis}}\xspace}


%Usando tikz
\usetikzlibrary{positioning}
%\usepgflibrary{decorations.pathreplacing}
\usetikzlibrary{decorations.pathreplacing}
\usetikzlibrary{decorations.pathmorphing}
\usetikzlibrary[patterns]
%\tikzstyle{every text node part}
%\usetikzlibrary{arrows,backgrounds,positioning,fit} 
\usetikzlibrary{calc}

% para gerar graficos no latex
\usepackage{pgfplots}
\pgfplotsset{compat=newest}

%Importando alguns pacotes
\usepackage[newfloat]{minted}
\usepackage{caption}
\usemintedstyle{borland}
\usepackage{amsfonts}
\usepackage{amssymb}
\usepackage{amsmath}
\usepackage{MnSymbol}
\usepackage[brazil]{babel}
\usepackage[utf8]{inputenc}
\usepackage[Algoritmo]{algorithm}
\usepackage[noend]{algorithmic}

\usepackage[brazilian,hyperpageref]{backref}	 % Paginas com as citações na bibl
\usepackage[alf]{abntex2cite}	% Citações padrão ABNT

%Setando bibliografia beamer
\setbeamertemplate{bibliography entry title}{}
\setbeamertemplate{bibliography entry location}{}
\setbeamertemplate{bibliography entry note}{}

\newcounter{saveenumi}
\newcommand{\seti}{\setcounter{saveenumi}{\value{enumi}}}
\newcommand{\conti}{\setcounter{enumi}{\value{saveenumi}}}

%\usepackage{shadethm}

%\definecolor{shadethmcolor}{rgb}{.75,.75,.75}

%Definindo novas cores
\definecolor{LightGray}{gray}{0.95}
\definecolor{shadecolor}{rgb}{.9, .9, .9}

%\newshadetheorem{theorem}{\scshape Teorema}[chapter]
\newtheorem{teorema}[theorem]{\scshape Teorema}
\newtheorem{proposicao}[theorem]{\scshape Proposição}
\newtheorem{corolario}[theorem]{\scshape Corolário}
\newtheorem{lema}[theorem]{\scshape Lema}
\newtheorem{definicao}[theorem]{\scshape Definição}
\newtheorem{conjectura}[theorem]{\scshape Conjectura}
\newtheorem{escolio}[theorem]{\scshape Escólio}
\newtheorem{exemplo}[theorem]{\scshape Exemplo}
\newtheorem{exemplos}[theorem]{\scshape Exemplos}
\newtheorem{propriedade}[theorem]{\scshape Propriedade}

\renewcommand{\u}{{\bf u}}
\renewcommand{\v}{{\bf v}}
\renewcommand{\sin}{\operatorname{sen}}
\renewcommand{\tan}{\operatorname{tg}}
\providecommand{\cas}{\operatorname{cas}}
\providecommand{\mdc}{\mathrm{mdc}}
\providecommand{\f}{{\bf f}}

\newcommand{\ie}{\textit{i.e.}}
\newcommand{\eg}{\textit{e.g.}}
%\newcommand{\qed}{\hfill $\square$}

\renewcommand\Re{\operatorname{Re}}
\renewcommand\Im{\operatorname{Im}}

\providecommand{\x}{{\bf x}}
\providecommand{\y}{{\bf y}}
\providecommand{\w}{{\bf w}}
\providecommand{\f}{{\bf f}}
\providecommand{\q}{{\bf q}}
\providecommand{\bfa}{{\bf a}}
\providecommand{\bfb}{{\bf b}}
\providecommand{\bfc}{{\bf c}}
\providecommand{\bfd}{{\bf d}}
\providecommand{\bfe}{{\bf e}}
\providecommand{\bfs}{{\bf s}}
\providecommand{\bfz}{{\bf z}}
\providecommand{\zero}{{\bf 0}}
\providecommand{\spn}{\mathrm{span}}
\providecommand{\posto}{\mathrm{posto}}
\providecommand{\nul}{\mathrm{nul}}
\providecommand{\proj}{\mathrm{proj}}
\providecommand{\tr}{\mathrm{tr}}
\providecommand{\sgn}{\mathrm{sgn}}

\providecommand{\cov}{\mathrm{cov}}

\providecommand{\dilation}{\mathcal{D}}
\providecommand{\erosion}{\mathcal{E}}
\providecommand{\open}{\mathcal{O}}
\providecommand{\close}{\mathcal{C}}

\newcommand*{\Bhat}{\skew{3}{\hat}{B}}

%\usetheme{AnnArbor}
%\usetheme{Antibes}
%\usetheme{Bergen}
%\usetheme{Berkeley}
%\usetheme{Berlin}
%\usetheme{Boadilla}
%\usetheme{boxes}
%\usetheme{CambridgeUS}
%\usetheme{Copenhagen}
%\usetheme{Darmstadt}
%\usetheme{default}
%\usetheme{Frankfurt}
%\usetheme{Goettingen}
%\usetheme{Hannover}
%\usetheme{Ilmenau}
%\usetheme{JuanLesPins}
%\usetheme{Luebeck}
%\usetheme{Madrid}
%\usetheme{Malmoe}
%\usetheme{Marburg}
%\usetheme{Montpellier}
%\usetheme{PaloAlto}
%\usetheme{Pittsburgh}
%\usetheme{Rochester}
%\usetheme{Singapore}
%\usetheme{Szeged}
%\usetheme{Warsaw}
\usetheme{metropolis}

% copiado do site:
% http://latex-beamer-class.10966.n7.nabble.com/Covering-images-transparent-i-e-dimmed-figures-td1504.html
\usepackage{ifthen}

\makeatletter
\newcommand{\includecoveredgraphics}[2][]{
    \ifthenelse{\the\beamer@coveringdepth=1}{
        \tikz
            \node[inner sep=0pt,outer sep=0pt,opacity=0.15]
                {\includegraphics[#1]{#2}};
    }{
        \tikz
            \node[inner sep=0pt,outer sep=0pt]
                {\includegraphics[#1]{#2}};%
    }
} 
\makeatother


% Definindo novas cores para utilizar código
\definecolor{verde}{rgb}{0,0.5,0}
% Configurando layout para mostrar codigos C++
\usepackage{listings}
\lstset{
  language=C++,
  basicstyle=\ttfamily\small,
  keywordstyle=\color{blue},
  stringstyle=\color{verde},
  commentstyle=\color{red},
  extendedchars=true,
  showspaces=false,
  showstringspaces=false,
  numbers=left,
  numberstyle=\tiny,
  breaklines=true,
  backgroundcolor=\color{green!10},
  breakautoindent=true,
  captionpos=b,
  xleftmargin=5pt,
}

\newenvironment{code}{\captionsetup{type=listing}}{}
\SetupFloatingEnvironment{listing}{name=Código Fonte}

%%%%%%%%%%%%%%%%%%%%%%%%%%%%%%%%%%%%%%%%%%%%%%%%%%%%%%%%
%Definindo informações da apresentação                 %
%%%%%%%%%%%%%%%%%%%%%%%%%%%%%%%%%%%%%%%%%%%%%%%%%%%%%%%%

%Definindo Título
\title{Apresentação de TCC}
%\subtitle{Um modelo de apresentação}
%Definindo Instituição
\institute{Curso de Engenharia da Computação - UEMG - Unidade Divinópolis}
%Definindo autores
\author{Tiago Alves de Oliveira \and \\
Orientador: Nome do Orientador}
%Definindo data (pode ser colocada manualmente
\date{\today}
%Colocando logo da UEMG
\titlegraphic{\hfill\includegraphics[height=2cm]{figuras/UEMG_vertical.jpg}}

\begin{document}



%Criando primeiro slide
\maketitle


%Criando Sumário
\begin{frame}{sumario}
    \frametitle{Sumário}
    \tableofcontents
\end{frame}

\section{Introdução}

\section{Objetivos}

\section{Metodologia ou Matériais e Métodos}

\section{Desenvolvimento}

\section{Resultados e Análises}

\section{Conclusão}

\begin{frame}[label=conclusao]{Conclusão}
    Conclusão I
    
    Conclusão II
\end{frame}

\begin{frame}[label=trabalhosfuturos]{Trabalhos Futuros}
    Texto texto texo
    \begin{enumerate}
        \item<+-> {} 
        \item<+-> {}
        \item<+-> {}
        \item<+-> {}
    \end{enumerate}
\end{frame}

\section{Exemplos de utilização}

\begin{frame}[fragile]{Duas colunas}
     \begin{columns}[T] % contents are top vertically aligned
     \begin{column}[T]{5cm} % each column can also be its own environment
     Conteúdo da primeira colunas \\ dividido em duas linhas
     \end{column}
     \begin{column}[T]{5cm} % alternative top-align that's better for graphics
          \includegraphics[height=3cm]{figuras/UEMG_vertical.jpg}
     \end{column}
     \end{columns}
\end{frame}

\begin{frame}
  \frametitle{Blocos}
  \begin{block}{Bloco Padrão}
        Block content.
      \end{block}

      \begin{alertblock}{Bloco de Alerta}
        Block content.
      \end{alertblock}

      \begin{exampleblock}{Bloco de Exemplo}
        Block content.
      \end{exampleblock}
\end{frame}

\begin{frame}
    \frametitle{Utilização de Listas}
    Exemplo de listas normal aparecendo todos os itens
    \begin{enumerate}
        \vfill \item {C}
        \vfill \item {C++}
        \vfill \item {PHP}
        \vfill \item {JAVA}
        \vfill \item {C\#}
    \end{enumerate}
\end{frame}

\begin{frame}
    \frametitle{Utilização de Listas}
    Exemplo de listas normal aparecendo todos os itens
    \begin{enumerate}
       \vfill \item<+-> {C}
        \vfill \item<+-> {C++}
        \vfill \item<+-> {PHP}
        \vfill \item<+-> {JAVA}
        \vfill \item<+-> {C\#}
    \end{enumerate}
\end{frame}

\begin{frame}
    \frametitle{Utilização de Listas}
    Exemplo de listas normal aparecendo todos os itens
    \begin{itemize}
        \item {C}
        \item {C++}
        \item {PHP}
        \item {JAVA}
        \item {C\#}
    \end{itemize}
\end{frame}

\begin{frame}
    \frametitle{Utilização de Listas}
    Exemplo de listas uma por vez
    \begin{itemize}
        \vfill \item<+-> {C}
        \vfill \item<+-> {C++}
        \vfill \item<+-> {PHP}
        \vfill \item<+-> {JAVA}
        \vfill \item<+-> {C\#}
    \end{itemize}
\end{frame}

\begin{frame}
  \frametitle{Utlizando Tabelas}
  \begin{table}
    \caption{Linguagens de Programação mais utilizadas em 2017 \cite{ieeespectrum}}
    \begin{tabular}{lr}
      \toprule
      Linguagem & Ranking Spectrum\\
      \midrule
      Python & 100\% \\
      C & 100\% \\
      Java & 99,8\% \\
      C++ & 96,9\% \\
      C\# & 88,6\% \\
      \bottomrule
    \end{tabular}
  \end{table}
\end{frame}

\begin{frame}[fragile]{}
    \frametitle{Inserindo imagens}
    \justifying
     A inserção de figuras é realizada com o comando: \\
     \verb+\begin{figure}[pos]+ \\
     e a inserção de tabelas com o comando : \\
     \verb+\begin{table}[pos]+. \\
     O parâmetro pos pode ser uma combinação de:
    
    \begin{itemize}
        \item[h] No lugar onde ocorreu no texto;
        \item[t] No topo de uma página;
        \item[b] Na parte inferior de uma página;
        \item[p] Em uma página própria.
    \end{itemize}
\end{frame}

\begin{frame}[fragile]{}
    \frametitle{Inserindo codigo}
    \begin{minted}{c}
        int main(){
            printf("Hello World");
            return 0;
        }
    \end{minted}
\end{frame}


\begin{frame}[fragile]{}
    \frametitle{Inserindo codigo}
    \begin{minted}[frame=lines,framesep=2mm,baselinestretch=1.2,bgcolor=LightGray,
    fontsize=\footnotesize,linenos]{python}
        import numpy as np
        
        def soma(x, y):
            return x + y
    \end{minted}
\end{frame}

\begin{frame}[fragile]{}
    \frametitle{Inserindo arquivo com código}
    \inputminted{c}{main.c}
\end{frame}

\begin{frame}[fragile]{Inserindo arquivo com código}
    \begin{code}
        \captionof{listing}{Meu código em  C}
        \label{code:c-code}
        \inputminted[frame=lines,framesep=2mm,baselinestretch=1.2,bgcolor=LightGray,fontsize=\footnotesize,linenos]{c}{main.c}
       
    \end{code}
\end{frame}

\begin{frame}[fragile]{}
    \frametitle{Brincando com cores}
    \begin{minted}[bgcolor=cyan!10]{java}
        /**
        * comentario
        */
        public class HelloWorldApp {
          public static void main (String argv[])
          {
          // Comentario
          System.out.println("Hello World!");
          }
        }
    \end{minted}
    
\end{frame}


\begin{frame}[fragile]{}
    \frametitle{Adicionando arquivos (outro formato)}
    \lstinputlisting[language=c]{file.c}
    \begin{alertblock}<+->{Declaração de arquivo}
        Esquecer o * irá acarretar em um erro de execução.
    \end{alertblock}
\end{frame}

\begin{frame}[fragile]{Leitura um arquivo texto}
    \begin{itemize}
        \vfill \item<+-> Para fazer a leitura de um arquivo texto é necessário utilizar a função fopen, como pode ser visto no exemplo abaixo:
    \end{itemize}  
    \lstinputlisting[language=c]{open.c}
\end{frame}

\defverbatim[colored]
\lstI{
  \begin{lstlisting}[caption=Escrita de Arquivos, label=code1]
    int main() {
        // Define variables at the beginning
        // of the block, as in C:
        CStash intStash, stringStash;
        int i;
        char* cp;
        ifstream in;
        string line;
        [...]
    }
  \end{lstlisting}
}

\begin{frame}[fragile]{A Listings Demo}{C++}
    \lstI
\end{frame}



\begin{frame}[standout]
  Perguntas?
\end{frame}

\begin{frame}
  Obrigado!
  %Agradecimentos
\end{frame}

\begin{frame}{Bibliografia}
    \bibliography{bibliografia}
\end{frame}

\end{document}